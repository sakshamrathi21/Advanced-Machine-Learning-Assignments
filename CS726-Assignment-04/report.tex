\documentclass[a4paper,12pt]{article}
\usepackage{xcolor}
\usepackage{algorithm}
\usepackage{algorithm}
\usepackage{algpseudocode}
\usepackage{amsmath,amsfonts,amssymb}
\usepackage{url}
\usepackage{geometry}
\usepackage{fancyhdr}
\usepackage{graphicx}
\usepackage{hyperref}
\usepackage{subcaption}
\usepackage{caption}
\usepackage{titlesec}
\usepackage{tikz}
\usepackage{booktabs}
\usepackage{array}
\usetikzlibrary{positioning,arrows.meta} % Load the positioning library

\usetikzlibrary{shadows}
\usepackage{tcolorbox}
\usepackage{float}
\usepackage{lipsum}
\usepackage{mdframed}
\usepackage{pagecolor}
\usepackage{mathpazo}   % Palatino font (serif)
\usepackage{microtype}  % Better typography

% Page background color
\pagecolor{gray!10!white}

% Geometry settings
\geometry{margin=0.5in}
\pagestyle{fancy}
\fancyhf{}

% Fancy header and footer
\fancyhead[C]{\textbf{\color{blue!80}CS726 Programming Assignment -- 4 Report}}
\fancyhead[R]{\color{blue!80}Bayesian Bunch}
\fancyfoot[C]{\thepage}

% Custom Section Color and Format with Sans-serif font
\titleformat{\section}
{\sffamily\color{purple!90!black}\normalfont\Large\bfseries}
{\thesection}{1em}{}

% Custom subsection format
\titleformat{\subsection}
{\sffamily\color{cyan!80!black}\normalfont\large\bfseries}
{\thesubsection}{1em}{}

% Stylish Title with TikZ (Enhanced with gradient)
\newcommand{\cooltitle}[1]{%
  \begin{tikzpicture}
    \node[fill=blue!20,rounded corners=10pt,inner sep=12pt, drop shadow, top color=blue!50, bottom color=blue!30] (box)
    {\Huge \bfseries \color{black} #1};
  \end{tikzpicture}
}
\usepackage{float} % Add this package

\newenvironment{solution}[2][]{%
    \begin{mdframed}[linecolor=blue!70!black, linewidth=2pt, roundcorner=10pt, backgroundcolor=yellow!10!white, skipabove=12pt, skipbelow=12pt]%
        \textbf{\large #2}
        \par\noindent\rule{\textwidth}{0.4pt}
}{
    \end{mdframed}
}

% Document title
\title{\cooltitle{CS726 Programming Assignment -- 4 Report}}
\author{
\textbf{Saksham Rathi (22B1003)}\\
\textbf{Sharvanee Sonawane (22B0943)}\\
\textbf{Deeksha Dhiwakar (22B0988)}\\
\small Department of Computer Science, \\
Indian Institute of Technology Bombay \\}
\date{}

\begin{document}
\maketitle

\section*{Task 0: Environment Setup and Result Reproduction}
Here is how the model was loaded:
\begin{verbatim}
  model = EnergyRegressor(FEAT_DIM).to(DEVICE)
\end{verbatim}

And here is how the trained weights were loaded:
\begin{verbatim}
  model.load_state_dict(torch.load('../trained_model_weights.pth', map_location=DEVICE))
\end{verbatim}

Here is the output generated when we run the script:
\begin{verbatim}
  Using device: cuda

  --- Model Architecture ---
  EnergyRegressor(
    (net): Sequential(
      (0): Linear(in_features=784, out_features=4096, bias=True)
      (1): ReLU(inplace=True)
      (2): Linear(in_features=4096, out_features=2048, bias=True)
      (3): ReLU(inplace=True)
      (4): Linear(in_features=2048, out_features=1024, bias=True)
      (5): ReLU(inplace=True)
      (6): Linear(in_features=1024, out_features=512, bias=True)
      (7): ReLU(inplace=True)
      (8): Linear(in_features=512, out_features=256, bias=True)
      (9): ReLU(inplace=True)
      (10): Linear(in_features=256, out_features=128, bias=True)
      (11): ReLU(inplace=True)
      (12): Linear(in_features=128, out_features=64, bias=True)
      (13): ReLU(inplace=True)
      (14): Linear(in_features=64, out_features=32, bias=True)
      (15): ReLU(inplace=True)
      (16): Linear(in_features=32, out_features=16, bias=True)
      (17): ReLU(inplace=True)
      (18): Linear(in_features=16, out_features=8, bias=True)
      (19): ReLU(inplace=True)
      (20): Linear(in_features=8, out_features=4, bias=True)
      (21): ReLU(inplace=True)
      (22): Linear(in_features=4, out_features=2, bias=True)
      (23): ReLU(inplace=True)
      (24): Linear(in_features=2, out_features=1, bias=True)
    )
  )
  ------------------------
  
  Loading dataset from ../A4_test_data.pt...
  Dataset loaded in 0.17s. Shape: x=torch.Size([100000, 784]), energy=torch.Size([100000, 
  1])
  
  --- Test Results ---
  Loss: 288.1554
  --- Script Finished ---  
\end{verbatim}

As shown in the output above, the model was and dataset were loaded successfully. The model architecture is a feedforward neural network with 24 layers, and the dataset contains 100,000 samples. The loss value of 288.1554 indicates the performance of the model on the test dataset.

\section*{Task 1: MCMC Sampling Implementation}

\end{document}
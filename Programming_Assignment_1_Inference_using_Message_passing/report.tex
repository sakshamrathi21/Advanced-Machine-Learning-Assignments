\documentclass[a4paper,12pt]{article}
\usepackage{xcolor}
\usepackage{algorithm}
\usepackage{algorithm}
\usepackage{algpseudocode}
\usepackage{amsmath,amsfonts,amssymb}
\usepackage{geometry}
\usepackage{fancyhdr}
\usepackage{graphicx}
\usepackage{caption}
\usepackage{titlesec}
\usepackage{tikz}
\usepackage{booktabs}
\usepackage{array}
\usetikzlibrary{positioning,arrows.meta} % Load the positioning library

\usetikzlibrary{shadows}
\usepackage{tcolorbox}
\usepackage{float}
\usepackage{lipsum}
\usepackage{mdframed}
\usepackage{pagecolor}
\usepackage{mathpazo}   % Palatino font (serif)
\usepackage{microtype}  % Better typography

% Page background color
\pagecolor{gray!10!white}

% Geometry settings
\geometry{margin=0.5in}
\pagestyle{fancy}
\fancyhf{}

% Fancy header and footer
\fancyhead[C]{\textbf{\color{blue!80}CS726 Programming Assignment -- 1 Report}}
\fancyhead[R]{\color{blue!80}Bayesian Bunch}
\fancyfoot[C]{\thepage}

% Custom Section Color and Format with Sans-serif font
\titleformat{\section}
{\sffamily\color{purple!90!black}\normalfont\Large\bfseries}
{\thesection}{1em}{}

% Custom subsection format
\titleformat{\subsection}
{\sffamily\color{cyan!80!black}\normalfont\large\bfseries}
{\thesubsection}{1em}{}

% Stylish Title with TikZ (Enhanced with gradient)
\newcommand{\cooltitle}[1]{%
  \begin{tikzpicture}
    \node[fill=blue!20,rounded corners=10pt,inner sep=12pt, drop shadow, top color=blue!50, bottom color=blue!30] (box)
    {\Huge \bfseries \color{black} #1};
  \end{tikzpicture}
}
\usepackage{float} % Add this package

\newenvironment{solution}[2][]{%
    \begin{mdframed}[linecolor=blue!70!black, linewidth=2pt, roundcorner=10pt, backgroundcolor=yellow!10!white, skipabove=12pt, skipbelow=12pt]%
        \textbf{\large #2}
        \par\noindent\rule{\textwidth}{0.4pt}
}{
    \end{mdframed}
}

% Document title
\title{\cooltitle{CS726 Scribe Notes}}
\author{{\bf Saksham (22B1003), Sharvanee (22B0943), Deeksha (22B0988)} \\
\small Department of Computer Science, \\
Indian Institute of Technology Bombay \\}
\date{}

\begin{document}
\maketitle



\section{Marginal Probability}
Here is the pseudocode for sharing messages between the maximal cliques of the graph.

Firstly, we show how to calculate the Z value for the given graph.

\begin{algorithm}
    \caption{Computation of Partition Function \( Z \)}
    \begin{algorithmic}[1]
    \Require Graphical Model with maximal cliques and potentials
    \Ensure Partition function \( Z \)
    
    \State Construct the junction tree \( JT \) from maximal cliques
    \State Initialize adjacency list \( JT_{adj} \) from \( JT \)
    
    \State Select a root clique \( C_{root} \)
    \State Initialize depth map with \( C_{root} \) at depth 0
    
    \Function{DFS}{$node, parent, depth$}
        \For{each child in \( JT_{adj}[node] \)}
            \If{child $\neq$ parent}
                \State Update depth map
                \State Call DFS on child with depth $+1$
            \EndIf
        \EndFor
    \EndFunction
    
    \State Perform DFS from \( C_{root} \)
    
    \Function{SendMessage}{$C_{from}, C_{to}$}
        \State Compute separator set \( S = C_{from} \cap C_{to} \)
        \State Initialize message vector \( M \) of size \( 2^{|S|} \)
        \State Modify clique potential based on incoming messages
        \For{each state assignment in \( C_{from} \)}
            \State Compute corresponding separator index
            \State Aggregate message value
        \EndFor
        \State Store message \( M(C_{from} \to C_{to}) \)
    \EndFunction
    
    \State Initialize messages dictionary
    \State Initialize clique potentials
    
    \For{each clique from deepest to root}
        \State Send messages to parent cliques
    \EndFor
    
    \For{each clique from root to leaves}
        \State Send messages to child cliques
    \EndFor
    
    \State Compute partition function \( Z \) using root clique potential and received messages
    \State \Return \( Z \)
    
    \end{algorithmic}
\end{algorithm}
    



\end{document}
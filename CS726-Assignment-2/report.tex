\documentclass[a4paper,12pt]{article}
\usepackage{xcolor}
\usepackage{algorithm}
\usepackage{algorithm}
\usepackage{algpseudocode}
\usepackage{amsmath,amsfonts,amssymb}
\usepackage{geometry}
\usepackage{fancyhdr}
\usepackage{graphicx}
\usepackage{subcaption}
\usepackage{caption}
\usepackage{titlesec}
\usepackage{tikz}
\usepackage{booktabs}
\usepackage{array}
\usetikzlibrary{positioning,arrows.meta} % Load the positioning library

\usetikzlibrary{shadows}
\usepackage{tcolorbox}
\usepackage{float}
\usepackage{lipsum}
\usepackage{mdframed}
\usepackage{pagecolor}
\usepackage{mathpazo}   % Palatino font (serif)
\usepackage{microtype}  % Better typography

% Page background color
\pagecolor{gray!10!white}

% Geometry settings
\geometry{margin=0.5in}
\pagestyle{fancy}
\fancyhf{}

% Fancy header and footer
\fancyhead[C]{\textbf{\color{blue!80}CS726 Programming Assignment -- 2 Report}}
\fancyhead[R]{\color{blue!80}Bayesian Bunch}
\fancyfoot[C]{\thepage}

% Custom Section Color and Format with Sans-serif font
\titleformat{\section}
{\sffamily\color{purple!90!black}\normalfont\Large\bfseries}
{\thesection}{1em}{}

% Custom subsection format
\titleformat{\subsection}
{\sffamily\color{cyan!80!black}\normalfont\large\bfseries}
{\thesubsection}{1em}{}

% Stylish Title with TikZ (Enhanced with gradient)
\newcommand{\cooltitle}[1]{%
  \begin{tikzpicture}
    \node[fill=blue!20,rounded corners=10pt,inner sep=12pt, drop shadow, top color=blue!50, bottom color=blue!30] (box)
    {\Huge \bfseries \color{black} #1};
  \end{tikzpicture}
}
\usepackage{float} % Add this package

\newenvironment{solution}[2][]{%
    \begin{mdframed}[linecolor=blue!70!black, linewidth=2pt, roundcorner=10pt, backgroundcolor=yellow!10!white, skipabove=12pt, skipbelow=12pt]%
        \textbf{\large #2}
        \par\noindent\rule{\textwidth}{0.4pt}
}{
    \end{mdframed}
}

% Document title
\title{\cooltitle{CS726 Programming Assignment -- 2 Report}}
\author{
\textbf{Saksham Rathi (22B1003)}\\
\textbf{Sharvanee Sonawane (22B0943)}\\
\textbf{Deeksha Dhiwakar (22B0988)}\\
\small Department of Computer Science, \\
Indian Institute of Technology Bombay \\}
\date{}

\begin{document}
\maketitle

\section*{Denoising Diffusion Probabilistic Models}

Here are the results of unconditional DDPMs on various datasets (with respect to the number of time steps). We had fixed all other parameters (the best settings observed):

\begin{itemize}
  \item lbeta=0.0001
  \item ubeta=0.02
  \item lr=0.0001 (so that training loss decreases across epochs)
  \item n\_samples=10000
  \item n\_dim=2 (for helix it is 3)
  \item batch\_size=128 (to avoid CUDA memory errors and optimal results)
  \item epochs=40
\end{itemize}

\clearpage
\subsection*{Moons}

\begin{figure}[h]
  \centering
  \begin{minipage}{0.3\textwidth}
      \centering
      \includegraphics[width=\linewidth]{images/moon.jpg}
      \subcaption{Original Moons Dataset}
  \end{minipage}
  \begin{minipage}{0.3\textwidth}
      \centering
      \includegraphics[width=\linewidth]{"images/Samples for ddpm_2_10_0.0001_0.02_moons.png"}
      \subcaption{Number of time steps = 10}
  \end{minipage}
  \begin{minipage}{0.3\textwidth}
      \centering
      \includegraphics[width=\linewidth]{"images/Samples for ddpm_2_50_0.0001_0.02_moons.png"}
      \subcaption{Number of time steps = 50}
  \end{minipage}

  \vspace{0.5cm}

  \begin{minipage}{0.3\textwidth}
      \centering
      \includegraphics[width=\linewidth]{"images/Samples for ddpm_2_100_0.0001_0.02_moons.png"}
      \subcaption{Number of time steps = 100}
  \end{minipage}
  \begin{minipage}{0.3\textwidth}
      \centering
      \includegraphics[width=\linewidth]{"images/Samples for ddpm_2_150_0.0001_0.02_moons.png"}
      \subcaption{Number of time steps = 150}
  \end{minipage}
  \begin{minipage}{0.3\textwidth}
      \centering
      \includegraphics[width=\linewidth]{"images/Samples for ddpm_2_200_0.0001_0.02_moons.png"}
    \subcaption{Number of time steps = 200}
  \end{minipage}

  \caption{Moons Dataset}
\end{figure}

Here are the NLL values:
\begin{itemize}
  \item $T = 10$: 1.048
  \item $T = 50$: 0.9599
  \item $T = 100$: 0.9519
  \item $T = 150$: 0.9218
  \item $T = 200$: 0.9321
\end{itemize}

As, we can see from both NLL values and the images, $T = 150$ performed the best.

\clearpage



\subsection*{Blobs}

\begin{figure}[h]
  \centering
  \begin{minipage}{0.3\textwidth}
      \centering
      \includegraphics[width=\linewidth]{images/blobs.png}
      \subcaption{Original Blobs Dataset}
  \end{minipage}
  \begin{minipage}{0.3\textwidth}
      \centering
      \includegraphics[width=\linewidth]{"images/Samples for ddpm_2_10_0.0001_0.02_blobs.png"}
      \subcaption{Number of time steps = 10}
  \end{minipage}
  \begin{minipage}{0.3\textwidth}
      \centering
      \includegraphics[width=\linewidth]{"images/Samples for ddpm_2_50_0.0001_0.02_blobs.png"}
      \subcaption{Number of time steps = 50}
  \end{minipage}

  \vspace{0.5cm}

  \begin{minipage}{0.3\textwidth}
      \centering
      \includegraphics[width=\linewidth]{"images/Samples for ddpm_2_100_0.0001_0.02_blobs.png"}
      \subcaption{Number of time steps = 100}
  \end{minipage}
  \begin{minipage}{0.3\textwidth}
      \centering
      \includegraphics[width=\linewidth]{"images/Samples for ddpm_2_150_0.0001_0.02_blobs.png"}
      \subcaption{Number of time steps = 150}
  \end{minipage}
  \begin{minipage}{0.3\textwidth}
      \centering
      \includegraphics[width=\linewidth]{"images/Samples for ddpm_2_200_0.0001_0.02_blobs.png"}
    \subcaption{Number of time steps = 200}
  \end{minipage}

  \caption{Blobs Dataset}
\end{figure}

Here are the NLL values:
\begin{itemize}
  \item $T = 10$: 0.37
  \item $T = 50$: 0.0152
  \item $T = 100$: 0.0232
  \item $T = 150$: -0.0223
  \item $T = 200$: 0.0045
\end{itemize}

As, we can see from both NLL values and the images, $T = 150$ performed the best. Moreover, there is a sudden decrease in NLL from 10 to 50, which shows the significant impact of increasing the number of time steps.

\clearpage
\subsection*{Many-Circles}

\begin{figure}[h]
  \centering
  \begin{minipage}{0.3\textwidth}
      \centering
      \includegraphics[width=\linewidth]{images/manycircles.png}
      \subcaption{Original ManyCircles Dataset}
  \end{minipage}
  \begin{minipage}{0.3\textwidth}
      \centering
      \includegraphics[width=\linewidth]{"images/Samples for ddpm_2_10_0.0001_0.02_manycircles.png"}
      \subcaption{Number of time steps = 10}
  \end{minipage}
  \begin{minipage}{0.3\textwidth}
      \centering
      \includegraphics[width=\linewidth]{"images/Samples for ddpm_2_50_0.0001_0.02_manycircles.png"}
      \subcaption{Number of time steps = 50}
  \end{minipage}

  \vspace{0.5cm}

  \begin{minipage}{0.3\textwidth}
      \centering
      \includegraphics[width=\linewidth]{"images/Samples for ddpm_2_100_0.0001_0.02_manycircles.png"}
      \subcaption{Number of time steps = 100}
  \end{minipage}
  \begin{minipage}{0.3\textwidth}
      \centering
      \includegraphics[width=\linewidth]{"images/Samples for ddpm_2_150_0.0001_0.02_manycircles.png"}
      \subcaption{Number of time steps = 150}
  \end{minipage}
  \begin{minipage}{0.3\textwidth}
      \centering
      \includegraphics[width=\linewidth]{"images/Samples for ddpm_2_200_0.0001_0.02_manycircles.png"}
    \subcaption{Number of time steps = 200}
  \end{minipage}

  \caption{Many Circles Dataset}
\end{figure}

Here are the NLL values:
\begin{itemize}
  \item $T = 10$: 0.75
  \item $T = 50$: 0.548
  \item $T = 100$: 0.545
  \item $T = 150$: 0.558
  \item $T = 200$: 0.522
\end{itemize}

As, we can see from both NLL values and the images, $T = 200$ performed the best.

\clearpage

\subsection*{Circles}

\begin{figure}[h]
  \centering
  \begin{minipage}{0.3\textwidth}
      \centering
      \includegraphics[width=\linewidth]{images/circles.png}
      \subcaption{Original Circles Dataset}
  \end{minipage}
  \begin{minipage}{0.3\textwidth}
      \centering
      \includegraphics[width=\linewidth]{"images/Samples for ddpm_2_10_0.0001_0.02_circles.png"}
      \subcaption{Number of time steps = 10}
  \end{minipage}
  \begin{minipage}{0.3\textwidth}
      \centering
      \includegraphics[width=\linewidth]{"images/Samples for ddpm_2_50_0.0001_0.02_circles.png"}
      \subcaption{Number of time steps = 50}
  \end{minipage}

  \vspace{0.5cm}

  \begin{minipage}{0.3\textwidth}
      \centering
      \includegraphics[width=\linewidth]{"images/Samples for ddpm_2_100_0.0001_0.02_circles.png"}
      \subcaption{Number of time steps = 100}
  \end{minipage}
  \begin{minipage}{0.3\textwidth}
      \centering
      \includegraphics[width=\linewidth]{"images/Samples for ddpm_2_150_0.0001_0.02_circles.png"}
      \subcaption{Number of time steps = 150}
  \end{minipage}
  \begin{minipage}{0.3\textwidth}
      \centering
      \includegraphics[width=\linewidth]{"images/Samples for ddpm_2_200_0.0001_0.02_circles.png"}
    \subcaption{Number of time steps = 200}
  \end{minipage}

  \caption{Circles Dataset}
\end{figure}


Here are the NLL values:
\begin{itemize}
  \item $T = 10$: 1.081
  \item $T = 50$: 0.991
  \item $T = 100$: 0.9869
  \item $T = 150$: 1.004
  \item $T = 200$: 0.992
\end{itemize}


\clearpage

\subsection*{Helix}

\begin{figure}[h]
  \centering
  \begin{minipage}{0.3\textwidth}
      \centering
      \includegraphics[width=\linewidth]{images/helix.png}
      \subcaption{Original Helix Dataset}
  \end{minipage}
  \begin{minipage}{0.3\textwidth}
      \centering
      \includegraphics[width=\linewidth]{"images/Samples for ddpm_3_10_0.0001_0.02_helix_sigmoid.png"}
      \subcaption{Number of time steps = 10}
  \end{minipage}
  \begin{minipage}{0.3\textwidth}
      \centering
      \includegraphics[width=\linewidth]{"images/Samples for ddpm_3_50_0.0001_0.02_helix_sigmoid.png"}
      \subcaption{Number of time steps = 50}
  \end{minipage}

  \vspace{0.5cm}

  \begin{minipage}{0.3\textwidth}
      \centering
      \includegraphics[width=\linewidth]{"images/Samples for ddpm_3_100_0.0001_0.02_helix_sigmoid.png"}
      \subcaption{Number of time steps = 100}
  \end{minipage}
  \begin{minipage}{0.3\textwidth}
      \centering
      \includegraphics[width=\linewidth]{"images/Samples for ddpm_3_150_0.0001_0.02_helix_sigmoid.png"}
      \subcaption{Number of time steps = 150}
  \end{minipage}
  \begin{minipage}{0.3\textwidth}
      \centering
      \includegraphics[width=\linewidth]{"images/Samples for ddpm_3_200_0.0001_0.02_helix_sigmoid.png"}
    \subcaption{Number of time steps = 200}
  \end{minipage}

  \caption{Helix Dataset}
\end{figure}


Here are the NLL values:
\begin{itemize}
  \item $T = 10$: 1.6179
  \item $T = 50$: 1.514
  \item $T = 100$: 1.5198
  \item $T = 150$: 1.528
  \item $T = 200$: 1.528
\end{itemize}

As we can see from the images (and the NLL values), 50 performs the best.




\clearpage



\section*{Classifier-Free Guidance}

\section*{Reward Guidance}





\end{document}
